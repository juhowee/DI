% Dokumentin asetukset
\documentclass[12pt]{article}

\usepackage{mathtools}

\usepackage[finnish,english]{babel}		% Kielimäärittelyt



\usepackage{lmodern}					% Tavutus skandinaavisille
\usepackage[T1]{fontenc}
\usepackage[utf8]{inputenc}

% \usepackage{mathptmx}
	
\usepackage{tabularx}					% Paremmat taulukot

\usepackage{graphicx}
\usepackage{subcaption}					% Grafiikkapaketti

\usepackage{setspace} 					% Kansilehden tarkastajaboksia varten
\usepackage{hyphenat} 					% Joku höskä kansilehden toimintaa varten

% Plotit
\usepackage{pgfplots}

\usepackage{graphicx}


%Lähdeluettelon muotoilua
% Määritetään viittaukset nimi-vuosiluku -tyylille
\usepackage[backend=biber,style=authoryear,giveninits=true,maxcitenames=2]{biblatex}		
% Pakotetaan lähdeluettelon muotoilu sukunimi-etunimi -muotoon				
\DeclareNameAlias{author}{last-first}
% Lähdeluettelon merkintöjen välille 1.5 riviväliä
\setlength\bibitemsep{1.5\itemsep}
% Poistetaan lainausmerkit "" ja kursiivit teoksen nimistä lähdeluettelossa
\DeclareFieldFormat*{title}{#1}
\DeclareFieldFormat*{journaltitle}{#1}
\DeclareFieldFormat*{booktitle}{#1}

% Korvataan "ja" -> & -merkillä lähdeluettelossa
\AtBeginBibliography{
\renewcommand*{\finalnamedelim}{%
\ifnumgreater{\value{liststop}}{2}{}{}%
\addspace\&\space}%
}

% Korvataan "ja" -> & -merkillä viitteissä
\AtEveryCite{
\renewcommand*{\finalnamedelim}{%
\ifnumgreater{\value{liststop}}{2}{}{}%
\addspace\&\space}%
}

\usepackage[parfill]{parskip}			% Poistaa kappaleen ensimmäisen rivin sisennyksen käytöstä

\usepackage{helvet}						% Helvetica -fontti otsikoita varten

\usepackage{titlesec}					% Otsikkojen tyylien muokkaus
\usepackage[ampersand]{easylist}
\usepackage{amssymb}

\usepackage{fancyhdr}

\usepackage[textsize=small]{todonotes}	% TODO -merkinnät
\setlength{\marginparwidth}{2.3cm}

% Kuva- ja taulukkotekstien asetukset
\usepackage[labelsep=period]{caption}
\captionsetup{labelfont={it,bf},textfont=it}

% Listausmerkinnät
\ListProperties(Hide=100, Space*=0.6pt, Hang=true, Progressive=3ex, Style*=-- ,
Style2*=$\bullet$ ,Style3*=$\circ$ ,Style4*=\tiny$\blacksquare$ )

% Sivun asetukset
\RequirePackage[paper=a4paper,inner=4cm,outer=2cm,top=2.5cm,bottom=2.5cm,headsep=1em]{geometry}
%\setlength\brokenpenalty{1000}
\setstretch{1.2} % Line height

% Asetetaan uusi otsikko (section) alkamaan uudelta sivulta
\newcommand{\sectionbreak}{\clearpage}

% Sivunumerointi sivun oikeaan yläkulmaan
\pagestyle{myheadings}

% Sivunumerointityyli precontent -osiossa (roomalaiset)
\newenvironment{precontent}{\pagenumbering{roman}}{\cleardoublepage}

% Sivunumerointityyli content -osiossa (numeerinen)
\newenvironment{content}{\pagenumbering{arabic}}{}

% Sivunumerointi vain erillisellä komennolla 
\pagenumbering{gobble}

% Dokumentin tiedot
\title{TERÄSRAKENTEISEN HALLIN PITUUSSUUNTAISTEN JÄYKISTERAKENTEIDEN OPTIMOINTI}
\author{JUHO VASTAPUU}
\newcommand{\documenttype}{Diplomityö} % Työn tyyppi kansilehdellä, esim. "diplomityö"
\newcommand{\inspector}   {TkT Kristo Mela } % Työn tarkastaja
\newcommand{\approvaldate}{6.6.2016}      % Työn hyväksyttämispäivämäärä

% Sisällysluettelon otsikko suomeksi.
\addto\captionsfinnish{
  \renewcommand{\contentsname}%
    {SISÄLLYSLUETTELO}%
}

% ------------
% Images
%
% The first parameter is width of the image, and defaults to
% 70% of page width.
% Usage example: \centeredpicture[1.0]{esimKuva}{Matlabilla tehty PDF-muotoinen esimerkkikuva.}
% ------------
\newcommand{\centeredpicture}[3][0.7]{
        \begin{figure}[h]
        \begin{center}
        \includegraphics[width=#1\textwidth]{#2}
        \end{center}
        \caption{#3}
        \label{fig:#2}
        \end{figure}
}

\newcommand{\tablefromimage}[3][0.7]{
        \begin{table}[h]
        \caption{#3}
        \label{table:#2}
        \end{table}
        \begin{figure}[h]
        \begin{center}
        \includegraphics[width=#1\textwidth]{#2}
        \end{center}
        \end{figure}
}

% Poistaa () -merkinnät vuosiluvun ympäriltä lähdeluettelossa
%\renewcommand\harvardyearleft{\unskip, }
%\renewcommand\harvardyearright[1]{.} 



\addbibresource{biblio.bib}



\begin{document}

\selectlanguage{finnish}


\include{kansilehti}

% Määritetään precontent -osion otsikkotyyli
\titleformat{\section}
        {\Large\sffamily\bfseries}
        {\makebox[2em][l]{\thesection}}
        {0pt}
        {\uppercase}
\titlespacing{\section}{0pt}{0pt}{18pt}

\begin{precontent}
\section*{Tiivistelmä}
\begin{spacing}{1.0}
{\fontfamily{phv}\selectfont
\textbf{JUHO VASTAPUU}: Teräsrakenteisen hallin pituussuuntaisen jäykistysjärjestelmän optimointi\\
Tampereen teknillinen yliopisto\\
Diplomityö, x sivua, x liitesivua\\
Joulukuu 2018\\
Rakennustekniikan diplomi-insinöörin tutkinto-ohjelma\\
Pääaine: Rakennesuunnittelu\\
Tarkastaja: TkT Kristo Mela\par

Avainsanat: \par
}
\end{spacing}





\section*{Alkusanat}


\tableofcontents

\end{precontent}

% Määritetään content -osion otsikkotyylit
\titleformat{\section}
        {\Large\sffamily\bfseries}
        {\makebox[2em][l]{\thesection}}
        {0pt}
        {\uppercase}
\titlespacing{\section}{0pt}{0pt}{42pt}

\titleformat{\subsection}
        {\large\sffamily\bfseries}
        {\makebox[2em][l]{\thesubsection}}
        {0pt}
        {}
\titlespacing{\subsection}{0pt}{18pt}{12pt}

\titleformat{\subsubsection}
        {\large\sffamily\bfseries}
        {\makebox[3em][l]{\thesubsubsection}}
        {0pt}
        {}
\titlespacing{\subsubsection}{0pt}{18pt}{12pt}



\begin{content}


\section{Johdanto}

Tähän voisi ottaa esim kirjasta Structures (Schodek) rakenteiden mitoituksesta yleensä: serviceability, efficiency, construction, costs ja other.

\subsection{Tutkimuksen tausta}
\subsection{Työn tavoitteet}
\subsection{Työn rajaukset}

\section{Optimointi}
\subsection{Optimointitehtävän matemaattinen määrittely}

Matemaattisella optimoinnilla tarkoitetaan prosessia, jolla etsitään systemaattisesti asetetulle funktiolle paras mahdollinen arvo siten, että sille asetetut reunaehdot toteutuvat. Asettamalla optimoitava kohde sekä halutut rajoite-ehdot matemaattiseen muotoon, voidaan optimoimalla löytää matemaattisin keinoin paras käypä ratkaisu. Käyvällä ratkaisulla tarkoitetaan ratkaisua, joka kuuluu annettujen rajoite-ehtojen joukkoon. 

Matemaattisesti optimoinnissa on tavoitteena etsiä funktiolle käyvästä joukosta minimi- tai maksimiarvo. Optimointitekniikoita ja algoritmeja on kehitetty lukuisia ja kukin niistä soveltuu käytettäväksi eri tavalla eri optimointitehtäviin. Optimointi ja erilaisten optimoitimenetelmien tutkiminen on yksi matemaattisen operaatiotutkimuksen osa-alueista. Optimoinnista voidaan joissain yhteyksissä käyttää myös nimitystä matemaattinen ohjelmointi (mathematical programming), jolla viitataan matemaattisten algoritmien kehittämiseen ja ohjelmoimista optimointitarkoituksiin.  \parencite[1]{rao}

Optimointitehtävä kirjoitetaan matemaattisesti seuraavanlaisessa muodossa.
\begin{equation}
\label{optimointi_määrittely}
\text{Etsi } \textbf{x} = 
\begin{Bmatrix} 
x_1 \\ 
x_2 \\ 
\vdots \\
x_n  
\end{Bmatrix}
\text{    joka } \min_{x_1 \in S} f(\textbf{x}) \text{, siten että}
\end{equation}
\begin{equation}
\label{rajoitteet}
\begin{split}
g_i(\textbf{x}) \leq 0, \quad i = 1,2, \dots , m  \\ 
h_j(\textbf{x}) = 0, \quad j = 1,2, \dots , p
\end{split}
\end{equation}

missä \textbf{x} on vektori, joka sisältää n-kappaletta suunnittelumuuttujia, f(\textbf{x}) on tavoitefunktio, $g_i(\textbf{x})$ ja $h_j(\textbf{x})$ ovat rajoite-ehtoja ja S on tehtävän toteuttavien suunnittelumuuttujien muodostama joukko. Rajoite-ehdot voivat olla joko epäyhtälö- tai yhtälömuotoisesti ilmoitettuja. Suunnittelumuuttujien lukumäärä (n) sekä rajoite-ehtojen lukumäärä (m ja/tai p) eivät ole riippuvaisia toisistaan. Tällaista optimointitehtävää kutsutaan rajoitetuksi optimointiongelmaksi. Optimointiongelman ei kuitenkaan tarvitse olla rajoitettu, vaan se voidaan ilmoittaa myös rajoittamattomana. Kaavassa \eqref{rajoitteet} on esitetty optimointitehtävän standardimuotoinen asettelu (standard design optimization model). \parencite[6]{rao}

Vektori \textbf{x} sisältää optimointitehtävän kaikki suunnittelumuuttujat (design variables). Muuttamalla jonkin suunnittelumuuttujan $x_i$ arvoa, muuttuu myös tavoitefunktion f(\textbf{x}) arvo. Suunnittelumuuttujista voidaan käyttää myös nimitystä optimointimuuttujat tai vapaat muuttujat, eli niiden arvoja voidaan muutella vapaasti kun haetaan tavoitefunktiolle arvoa. Toisistaan riippumattomien eli itsenäisten suunnittelumuuttujien lukumäärä on optimointiongelman vapausasteluku (design degree of freedom). Yleisesti ottaen suunnittelumuuttujien tulee olla toisistaan riippumattomia, mutta joissain tapauksissa niiden määrä voi olla ongelman vapausastelukua suurempi. Tämä on perusteltua esimerkiksi silloin, kun kohdefunktion määrittely pelkillä itsenäisillä suunnittelumuuttujilla olisi hankalaa. Jokaiselle suunnittelumuuttujalle täytyy myös pystyä asettamaan jokin numeerinen lähtöarvo, jotta optimointitehtävä pystytään suorittamaan. 

Kohde- tai tavoitefunktio f(\textbf{x}) (objective function) on optimointitehtävän matemaattinen muoto ilmoitettuna suunnittelumuuttujavektorin \textbf{x} funktiona. Optimointitehtävän tavoitteena on joko minimoida tai maksimoida kohdefunktion arvo. Mikäli optimointitehtävässä on useampi kuin yksi kohdefunktio, käytetään tehtävästä nimitystä monitavoiteoptimointi (multiobjective design optimization). Tällöin kohdefunktio ilmaistaan matemaattisesti kohdefunktioiden joukkona 
\begin{align}
\textbf{f(x)} = \begin{bmatrix}
f_1(\textbf{x}) &  f_2(\textbf{x}) & \cdots & f_p(\textbf{x})
\end{bmatrix},
\end{align}
jossa jokainen kohdefunktio koostuu kuitenkin samasta suunnittelumuuttujavektorista \textbf{x}.

Optimoitavalle kohteelle asetettavat rajoite-ehdot esitetään rajoitefunktioina $g_i(\textbf{x})$ ja $h_j(\textbf{x})$. Optimointialgoritmi ratkaisee optimointitehtävän siten, että kohdefunktion arvo toteuttaa rajoite-ehdot. Rajoite-ehtojen muodostamaa joukkoa kutsutaan täten optimointiongelman käyväksi joukoksi (feasible region). Käypää joukkoa esittää kaavassa (\ref{optimointi_määrittely}) osoitettu joukko S, joka määritellään 

\begin{equation}
\label{käypä_joukko}
S = \{\textbf{x} | g_i(\textbf{x}) \leq 0; h_j(\textbf{x}) = 0; i = 1, 2, \dots,m; j = 1, 2, \dots,p\} 
\end{equation}

Mitä tahansa käyvässä joukossa $S$ olevaa kohdefunktion arvoa kutsutaan käyväksi ratkaisuksi (feasible design) huolimatta siitä onko kyseessä optimiratkaisu. Kaksiuloitteisessa tapauksessa käypää joukkoa voidaan havainnollistaa piirtämällä rajoitefunktioiden käyrät koordinaatistoon. Käypä joukko muodostuu näiden käyrien rajoittamana alueena. Käyvän joukon negaatiota kutsutaan ei-käyväksi joukoksi (unfeasible region). Sekä tavoitefunktion f(\textbf{x}), että rajoite-ehtojen $g_i(\textbf{x})$ ja $h_j(\textbf{x})$ on oltava toisitaan joko implisiittisesti tai eksplisiittisesti riippuvia. Mikäli riippuvuutta funktioiden välillä ei ole, ei optimointitehtävää voi muodostaa eikä varsinaista optimointiongelmaa voi edes osoittaa. \parencite[43]{arora} 

Kuten optimointitehtävän määrittelevä kaava \eqref{rajoitteet} osoittaa, optimointiongelmalle voidaan asettaa rajoite-ehtoja sekä yhtälö- että epäyhtälömuodossa. Epäyhtälörajoitteita kutsutaan käypään joukkoon nähden toispuoleisiksi rajoite-ehdoiksi (unilateral constraints tai one-sided constraints). Epäyhtälörajoitteiden rajoittama käypä joukko on täten paljon laajempi kun verrataan yhtälörajoitteista käypää joukkoa. Esimerkiksi kakisuloitteisessa tapauksessa yhtälörajoite tarkoittaisi, että käypä ratkaisu löytyisi rajoitefunktion käyrältä. \parencite[16-18]{arora} 

Yhtälömuotoisten rajoite-ehtojen määrän tulee olla maksimissaan suunnittelumuuttujien määrä, toisin sanottuna optimointitehtävän kaavan \eqref{rajoitteet} tulee toteuttaa ehto

\begin{equation}
\label{yhtälörajoite-ehto}
p \leq n.
\end{equation}

Tapaus, jossa yhtälömuotoisia rajoite-ehtoja on annettuja suunnittelumuuttujia enemmän, on kyseessä ylimääritetty yhtälöryhmä (overdetermined system). Tällaisessa tapauksessa rajoite-ehtojen joukossa on ylimääräisiä eli redundatteja ehtoja, jotka toteuttavat suoraan jonkun muun rajoite-ehdon, eikä niiden ilmoittaminen täten ole tarpeellista. Triviaalitapauksessa jossa suunnittelumuuttujien määrä ja yhtälömuotoisten rajoite-ehtojen määrä on yhtäsuuri, löytyy tehtävälle ratkaisu analyyttisin keinoin eikä optimointi ole tarpeellista. Kaksiuloitteisessa tapauksessa tämä tarkoittaisi kahden käyrän leikkauspistettä. 

Standardimuotoisessa optimointitehtävässä epäyhtälörajoitteet ilmoitetaan aina kaavan \eqref{rajoitteet} osoittamassa muodossa, eli siten että rajoite-ehto on pienempi tai yhtäsuuri kuin nolla ($\leq$ 0). Tästä huolimatta voidaan optimointitehtävässä käsitellä myös $\geq$ -tyyppisiä rajoite-ehtoja. Standardimuotoista tehtävää aseteltaessa nämä voidaan muuttaa $\leq$ -muotoon yksinkertaisesti kertomalla ehto luvulla -1. Epäyhtälömuotoisten rajoite-ehtojen määrää ei ole rajoitettu, toisin kuin yhtälömuotoiset rajoitteet kaavassa \eqref{yhtälörajoite-ehto}. Niiden määrää ei ole siis rajoitettu.   \parencite[43]{arora}

\subsection{Optimointitehtävän luokittelu}

Optimointitehtävän luokittelu perustuu tehtävän asettelun eli kohdefunktioiden, rajoite-ehtojen ja suunnittelumuuttujien matemaattiseen muotoon. Eri tyyppisille optimoinitehtäville on sovellettava erilaisia ratkaisumenetelmiä. Tämän vuoksi tehtävän oikeanlainen ja mahdollisimman tarkka luokittelu on tärkeää, jotta tehtävälle löytyisi mahdollisimman tehokas ratkaisukeino. Singresu Rao esittää kirjassaan \parencite{rao} kahdeksan optimointitehtävien jaottelutapaa. Sen mukaan optimointitehtävä voidaan luokitella:

\begin{enumerate}
\item Rajoitetuksi tai ei-rajoitetuksi
\item Staattiseksi tai dynaamiseksi
\item Kohdefunktion tai rajoite-ehtojen matemaattisen muodon perusteella
\item Säätöongelmaksi
\item Diskreetiksi tai jatkuvaksi tehtäväksi
\item Deterministiseksi tai stokastiseksi tehtäväksi
\item Kohdefunktion separoituvuuden perusteella
\item Kohdefunktioiden määrän perusteella
\end{enumerate}

Rajoitefunktioiden perusteella tehtävä voidaan luokitella joko rajoitetuksi- tai ei-rajoitetuksi tehtäväksi. Mikäli tehtävällä on yksikin rajoitefunktio, on kyseessä rajoitettu optimointitehtävä. Luokittelua voidaan tarkentaa edelleen osittain rajoitetuksi tai täysin suljetuksi systeemiksi. Suljetulla systeemillä tarkoitetaan tilannetta, jossa rajoitefunktiot muodostavat äärellisen kokoisen käyvän joukon. 

Suunnittelumuuttujien perusteella tehtävä voidaan jakaa staattiseksi tai dynaamiseksi tehtäväksi. Staattisessa tai parametrisessa tehtävässä kohdefunktio on määritelty suunnittelumuuttujien suhteessa ja tehtävänä on ratkaista suunnittelumuuttujat. Kaavan \eqref{rajoitteet} määrittelyssä kyseessä on staattinen optimointitehtävä, jossa siis etsitään suunnittelumuuttujille arvo siten, että se minimoi kohdefunktion. Dynaamisessa tehtävässä kohdefunktio puolestaan muodostuu funktioista, jotka on määritelty jonkun tietyn parametrin suhteen, kuten esimerkiksi seuraavasti.

\begin{equation}
\label{dynaaminen}
\text{Etsi } \textbf{x}(t) = 
\begin{Bmatrix} 
x_1(t) \\ 
x_2(t) \\ 
\vdots \\ 
x_n(t) \\ 
\end{Bmatrix}
\text{    joka minimoi } f[\textbf{x}(t)].
\end{equation}

Dynaamisessa tehtävässä etsitään siis kohdefunktioon sijoitettavien suunnittelumuuttujien sijasta funktiot, jotka esitetään jonkin parametrin suhteen. \parencite[15]{rao}

Optimointitehtävä luokitellaan myös kohdefunktion tai rajoite-ehtojen matemaattisen muodon perusteella. Tämä luokittelutapa on erityisen kriittinen optimointitehtävän ratkaisun kannalta, sillä monet optimointialgoritmit toimivat vain tietyntyyppisille optimointitehtäville juurikin kohdefunktioiden tai rajoite-ehtojen matemaattisen muodon mukaan. Yksi yleisin optimointitehtävän muoto on epälineaarinen ongelma (nonlinear programming problem, NLP). Optimointitehtävä on epälineaarinen mikäli sen kohdefunktio tai yksikin rajoitefunktioista on muodoltaan epälineaarinen. Optimoinnissa muodostuvat ongelmat ovat hyvin usein muodoltaan epälineaarisia ja matematiikan osa-aluetta, joka tutkii epälineaaristen optimointiongelmien ratkaisua, kutsutaan epälineaariseksi matemaattiseksi ohjelmoinniksi (nonlinear programming). Muita optimoinnin tehtävätyyppejä ovat esimerkiksi geometrinen ja kvadraattinen ongelma. \parencite[19]{rao} 

\todo{tarkista oikea suomenkielinen termi !}
Säätöongelmassa tai säätöoptimointitehtävässä (optimal control) on kyse tehtävästä, joka jakautuu useampiin osioihin tai sekvensseihin \todo{tarkista termin oikeellisuus}. Jokainen sekvenssi kehittyy edellisestään määritetyllä tavalla. Tehtävän määrittelyyn käytetään suunnittelumuuttujien lisäksi tilamuuttujia (state variables). Säätöoptimointitehtävässä suunnittelumuuttujat määrittävät systeemin kussakin sekvenssissä sekä tavan, jolla systeemi siirtyy seuraavaan sekvenssiin. Tilamuuttujat puolestaan määrittävät kussakin sekvenssissä systeemin tilan, eli tutkittavan ongelman käyttäytymisen kussakin sekvenssissä. Säätöongelmassa tehtävänä on löytää suunnittelu- tai tilamuuttujille sellaiset arvot kussakin eri sekvenssissä, että kohdefunktioiden summa eri sekvensseissä saadaan minimoitua rajoite-ehdot huomoiden. \parencite[19]{rao} Tämänkaltaiset tehtävät ovat tavallisia sellaisissa teknisissä sovelluksissa, joiden tila muuttuu jatkuvasti ja tilan ylläpitoon vaaditaan resursseja. Yksi yleinen säätöoptimointitehtäviä soveltava tekniikan ala on säätötekniikka.

Suunnittelumuuttujien saadessa vain diskreettejä arvoja, käytetään optimointitehtävästä nimitystä diskreetti tai ei-jatkuva tehtävä. Tämän kaltainen tehtävä voidaan yleistää kokonaislukuoptimoinniksi (integer programming problem). Tehtävän vastakohta on jatkuva tehtävä, jossa siis sallitaan kaikille suunnittelumuuttujille reaalilukuarvo (real-valued programming problem). \parencite[28]{rao} Insinööritieteissä suunnittelumuuttujat valitaan yleensä ennalta määritellystä joukosta käytössä olevien resurssien mukaan, eli käsiteltävät optimointitehtävät ovat usein diskreettejä. 

Suunnittelumuuttujat tai kohdefunktion parametrit voivat saada määriteltyjen eli deterministisien arvojen sijasta todennäköisyyteen perustuvia arvoja. Tällöin optimointitehtävä on stokastinen eli siinä käsitellään determinististen muuttujien sijasta satunnaismuuttujia. \parencite[29]{rao} \todo{tähän lisää}

Optimointitehtävä on separoituva, mikäli kohdefunktio ja rajoite-ehdot voidaan esittää yhden muuttujan funktioiden summana, eli muodossa 

\begin{equation}
\label{separoituva_funktio}
f(\textbf{x}) = \sum_{i=1}^n f_i(x_i)
\end{equation}.

Mikäli optimointitehtävä sisältää yhden tavoitefunktion sijasta useamman funktion, on kyseessä monitavoiteoptimointi. 




\subsection{Rakenteiden optimointi}

Rakenteiden optimoinnissa tavoitteena on löytää mahdollisimman taloudellinen, mutta vaatimukset täyttävä rakenne. Tavallisimpia optimointitehtäviä ovat esimerkiksi rakenteiden massan tai tuotantokustannusten minimointi. Kun suunniteltava rakenne asetetaan matemaattiseksi malliksi ja formuloidaan optimointitehtävän muotoon, voidaan rakenne optimoida.  

Useista sauvoista koostuvan rakennekokonaisuuden optimointi jaotellaan tavallisesti mitoitusoptimointiin, muodon eli geometrian optimointiin ja topologian optimointiin. Rakennekokonaisuuden geometrialla tarkoitetaan sen sisältämien sauvojen välisten solmupisteiden sijaintia suhteessa toisiinsa. Topologialla puolestaan tarkoitetaan kokonaisuuden sisältämien solmupisteiden määrää. Esimerkiksi teräsristikon topologia sisältää tiedon paarresauvojen ja diagonaalien välisten solmupisteiden määrästä, mutta ei niiden sijainnista \parencite[2]{mela_doct}.

Mitoitusoptimoinnissa rakennekokonaisuudelle haetaan sellaiset rakenneosat, joilla asetetut reunaehdot täyttyvät kuitenkaan rakennekokonaisuuden geometriaa muuttamatta. Esimerkiksi teräsristikon mitoitusoptimointitehtävässä haetaan määrätyn muotoisen ristikon kullekkin paarre- ja diagonaalisauvalle mahdollisimman pieni poikkileikkaus. Mitoitusoptimointitehtävä on siis tyypiltään hyvin yksinkertainen eikä välttämättä yksinkertaisilla rakenteilla vaadi kovinkaan monimutkaista laskentaa.
\begin{figure}[htb]

\includegraphics[width=7cm]{Kuvat/mitoitusoptimointi.pdf}
\caption{Havainne-esimerkki teräsputkiristikon mitoitusoptimoinnista.}
\label{fig:mitoitusoptimointi}
\end{figure}

Muodon tai geometrian optimoinnissa eri rakenneosien välisten liitoskohtien sijaintia muuttamalla pyritään hakemaan paras ratkaisu siten, että asetetut reunaehdot täyttyvät. Geometrian optimoinnissa rakenneosien välisten liitosten lukumäärä säilyy vakiona, ainoastaan niiden keskinäinen sijainti muuttuu. Tavallisesti tähän tehtävätyyppiin liitetään myös mitoitusoptimointi. Esimerkiksi eräsristikon geometrian optimointitehtävässä vaihdetaan diagonaalisauvojen liitoskohtien sijaintia paarteilla ja pyritään sillä hakemaan ristikolle optimirakenne.

Topologian optimoinnissa rakennekokonaisuudelle haetaan optimitopologia siten, että asetetut reunaehdot täyttyvät. Koska topologia käsitteenä kuvaa ainoastaan rakennekokonaisuuden sauvojen liittymistä toisiinsa sekä kokonaisuudessa olevien liitosten määrää, sisältää topologian optimointi tavallisesti myös geometrian ja mitoituksen optimoinnin. Topologian optimoinnissa haetaan siis esimerkiksi teräsristikon tapauksessa optimein sauvajärjestely muuttamalla sekä sauvojen ja liitosten määrää että liitosten välistä keskinäistä sijaintia. 






\section{Teräshallin jäykistäminen}

\subsection{Jäykistäminen yleisesti}

Kantavien rakenneosien stabiliteetti varmistetaan jäykistämällä rakennus. Rakennus on jäykistettava sekä vaaka- että pystysuunnassa ja jäykisteille kohdistuvat kuormat on vietävä perustuksille. 

Jäykistäminen ja jäykistejärjestelmän toimintatarkoitus jaetaan ulkoisia kuormia ja sisäisiä kuormia vastaanottaviin jäykisterakenteisiin. Tavallisesti jäykisterakenteista puhuttaessa ulkoisilla kuormilla tarkoitetaan ulkoisia vaakavoima, joita ovat esimerkiksi tuulikuorma ja pystykuormista rakenteille niiden epätarkkuuksista aiheutuvat lisävaakavoimat. Vaakakuormat aiheuttavat rakennukseen sivusiirtymiä tai kiertymää, jolloin pystykuormat muuttuvat rakenneosiin nähden epäkeskeiseksi ja puristetut rakenneosat ovat vaarassa menettää kantavuutensa esimerkiksi stabiliteetin menetyksen seurauksena. Jäykisterakenteiden on tarkoitus estää tätä vaakakuormista aiheutuvaa muodonmuutosta ja näin ollen säilyttää rakennuksen muoto jäykkänä kokonaisuutena, jolloin kuormat ohjautuvat rakenneosia pitkin suunnitellusti perustuksille. Tavallisia ulkoisia vaakakuormia vastaan toimivia jäykisterakenteita teräshallissa ovat katon tuuliristikko tai pilarien väliset vinositeet. 

Sisäisiä voimia vastaan toimivilla jäykisteillä tarkoitetaan rakenteita, jotka tukevat kantavan rungon puristettuja tai taivutettuja rakenneosia stabiliteetin menetystä vastaan. Puristetuilla ja myös taivutetuilla rakenneosilla mitoittavaksi tekijäksi usein muodostuu sisäiset stabiliteetti-ilmiöt, kuten nurjahdus ja kiepahdus. Sisäisen jäykistejärjestelmän tarkoitus on muodostaa tukipisteitä näihin rakenneosiin ja näin estää nämä stabiliteetinmenetysilmiöt rajoittamalla vapaata nurjahdus- tai kiepahduspituutta. Yksi esimerkki sisäisestä jäykistysjärjestelmästä teräshallissa on kattoristikon yläpaarteen nurjahdustuennat, joista tavallisesti käytetään yleisnimitystä katon jäykistesiteet. 



\subsection{Jäykisteen jäykkyyden ja lujuuden välinen yhteys}

Puristetun sauvan stabiliteetti perustuu Eulerin nurjahdukseen, jossa ideaalisuoraa homogeenisesta materiaalista koostuvaa hoikkaa sauvaa kuormitetaan keskeisesti. Kuormaa, joka saa aikaan sauvan stabiliteetin menetyksen kutsutaan kriittiseksi kuormaksi tai nurjahdusvoimaksi ja se määritetään Eulerin nurjahduksen kaavalla

\begin{equation}
\label{euler}
P_{cr} = \frac{\pi^2 E I}{L_{cr}^2},
\end{equation}

missä $E$ on materiaalin kimmokerroin, $I$ on poikkileikkauksen neliömomentti tarkasteltavan akselin ympäri ja $L_{cr}$ on kyseistä nurjahdusmuotoa vastaava nurjahduspituus. Kaavan (\ref{euler}) mukaan pienentämällä sauvan nurjahduspituutta $L$ kasvaa sauvan kriittinen kuorma eksponentiaalisesti. Nurjahduspituuden pienentäminen voidaan toteuttaa lisäämällä sauvan poikittaisten tukien määrää. Tätä toimenpidettä kutsutaan jäykistämiseksi. Jäykistämisellä pyritään kasvattamaan sauvan kuormankantokykyä parantamalla sen stabiliteettia. Yksinkertaista puristetun sauvan ja jäykisteen rakennetta on esitetty kuvassa (\ref{fig:puristettu_sauva}), jossa nivelellisesti tuettua kuormalla $P$ puristettua sauvaa tuetaan keskikohdastaan joustavalla jäykisterakenteella. Sauvalla on alkuepäkeskisyys $\Delta_0$ sekä jäykistesiteen puristumasta aiheutuva siirtymä $\Delta$. Jäykistesiteen puristuma aiheuttaa jäykisteeseen aksiaalisen puristusvoiman $P_{br}$.

\begin{figure}[htb]
\centering
\includegraphics[width=5cm]{Kuvat/puristettu_sauva.pdf}
\caption{Rakennemalliesimerkki jäykisteen toiminnasta, missä puristettua alkuepäkeskisyyden omaavaa sauvaa tuetaan jännevälin keskeltä jousimaisella jäykisterakenteella.}
\label{fig:puristettu_sauva}
\end{figure}

Jäykistävä rakenne tukee puristettua sauvaa siten, että nurjahdus pääsee tapahtumaan ainoastaan tuki- ja jäykistepisteiden välillä. Toimiakseen jäykisteellä tulee olla riittävä määrä aksiaalista jäykkyyttä sekä jäykisteen poikkileikkauksella lujuutta \parencite{winter}. Kun ideaalisuoraa sauvaa kuormitetaan kaavan (\ref{euler}) mukaisella kriittisellä kuormalla $P_{cr}$, on pienin toimivan jäykisteen aksiaalisen jäykkyyden arvo kaavan

\begin{equation}
\label{ideaalijäykkyys}
\beta_i = \frac{\gamma_i P_{cr}}{L},
\end{equation}

mukainen, missä $L$ on sauvaa tukipisteiden välinen etäisyys ja $\gamma_i$ on jäykistävien siteiden määrästä riippuva kerroin \parencite[76]{timoshenko}. Jäykistesiteen jäykkyys $\beta_i$ on siis pienin jäykkyys, jolla ideaalisuoran puristetun sauvan nurjahdus tapahtuu lokaalisti tukipisteiden välillä. $P_{cr}$ määritellään kaavan (\ref{euler}) mukaisesti käyttämällä nurjahduspituutena jäykisteiden välistä etäisyyttä. 

Kertoimen $\gamma_i$ avulla voidaan määrittää jäykisteiden ideaalijäykkyys, kun puristettua sauvaa tuetaan useammalla jäykisteellä ja kun jäykisteiden välinen keskinäinen etäisyys on vakio. Jäykkyyskertoimet ja niitä vastaavat rakenteet on esitetty taulukossa (\ref{tab:ideaalijäykkyydet}).

\begin{table}[htb]
\centering
\caption{Ideaalisuoraa sauvaa jäykistävien rakenteiden vaaditut jäykkyyden kertoimet.}
\begin{tabular}{c c c c c c}
\label{tab:ideaalijäykkyydet}

Siteiden määrä & 1 & 2 & 3 & 4 & 5   \\
\hline
Kerroin $\gamma_i$ & 2,000 &3,000 & 3,413 & 3,623 & 3,731 
\end{tabular}
\end{table}

Taulukosta huomataan, että siteiden määrän kasvaessa kasvaa myös vaadittu ideaalijäykkyyden arvo. Tämä on ilmeistä, sillä näin myös sauvaa voidaan kuormittaa suuremmalla puristusvoimalla sen nurjahduskestävyyden ollessa suurempi. Stephen Timoshenko esitti nämä kertoimet \parencite{timoshenko} johtamalla ne differentiaalilaskennan kautta jatkuvalle palkille joka tuettu jousituilla. 

Myös George Winter päätyi samoihin kertoimiin \parencite{winter} yksinkertaisemmalla tavalla määrittämällä tasapainoehdot sauvan ja jäykisteiden välisten liittymispisteiden ympäri ja ratkaisemalla kertoimet sauvan ominaismuotojen kautta. Winterin teoria perustui otaksumaan, jossa liitospisteessä on fiktiivinen nivel, jolloin sauvaan ei kohdistu lainkaan taivutusmomenttia. Winterin esittämä tasapainomalli ottaa huomioon myös sauvan alkuepäkeskisyyden $\Delta_0$. Kuvan (\ref{fig:puristettu_sauva}) rakennemallista voidaan vastaavalla tavalla muodostaa tasapainoehto

\begin{equation}
\label{tasapainoehto}
P (\Delta_0 + \Delta) = \frac{P_{br} L}{2},
\end{equation}

missä $P$ on sauvaan kohdistuva aksiaalinen voima, $\Delta_0$ sauvan alkuvinous, $\Delta$ jäykisteen puristuma ja $P_{br}$ on jäykisteen sauvaan kohdistama tukireaktio. Kun otaksutaan jäykisteessä vaikuttavan tukreaktion suuruisen aksiaalisen voiman, voidaan jousiteorian nojalla määrittää yhteys jäykkyyden ja puristuman välillä kaavalla

\begin{equation}
\label{jousiteoria}
P_{br} = \beta \Delta,
\end{equation}

missä $\beta$ on jäykistesauvan aksiaalinen jäykkyys. Kun sijoitetaan tämä rakenteen tasapainoehdon kaavaan (\ref{tasapainoehto}) saadaan aksiaalisen jäykkyyden arvoksi

\begin{equation}
\label{breq}
\beta_{req} = \frac{2 P}{L} (\frac{\Delta_0}{\Delta} + 1),
\end{equation}

missä $\beta_{req}$ on siis jäykisteenä toimivan rakenteen vaadittu aksiaalinen jäykkyys kun huomioidaan alkusiirtymä $\Delta_0$. Ideaalisuoralla sauvalla ($\Delta_0$ = 0) kaava saa saman arvon kuin aikaisemmin esitetty ideaalijäykkyyden arvo (\ref{ideaalijäykkyys}), sillä taulukon (\ref{tab:ideaalijäykkyydet}) mukaisesti kyseisen rakenteen jäykkyyskerroin $\gamma_i$ = 2,0. Voidaan siis kirjoittaa, että yleisesti

\begin{equation}
\label{breq_2}
\beta_{req} = \beta_i (\frac{\Delta_0}{\Delta} + 1).
\end{equation}






%Sauvan siirtyessä jäykisteen kohdalta kohdistaa se jäykisteeseen aksiaalisen puristuman. Jousiteorian nojalla jäykisteessä vaikuttava voima on 
%
%
%
%missä $\beta$ on jäykistesauvan aksiaalinen jäykkyys sekä $\Delta$ jäykistesauvan puristuma, eli puristetun sauvan siirtymä jäykistepisteessä. 
%
%Kun kirjoitetaan tasapainoyhtälö sauvan tukipisteen suhteen ottamalla huomioon sauvan alkuvinous $/Delta_0$, saadaan yhteys
%
%\begin{equation}
%\label{tasapainoehto}
%P (\Delta_0 + \Delta) = \beta L \Delta.
%\end{equation}
%
%Sijoittamalla tähän kaavan (\ref{jäykistevoima_jousiteoria}) mukainen sauvan puristuman ja jäykkyyden välinen yhteys, saadaan kuvan (\ref{fig:puristettu_sauva}) rakenteen jäykisteelle kohdistuvaksi jäykistevoiman arvoksi
%
%\begin{equation}
%\label{jäykistevoima}
%P_{br} = \frac{\beta P \Delta_0}{\beta L - P}.
%\end{equation}
%
%Alla olevissa kuvaajissa on havainnollistettu edellä esitettyjen kaavojen välisiä yhteyksiä. Kuvaajat on määritelty alkuvinouden arvolla $\Delta_0 = 0,002L$. 
%
%\begin{figure}[htb]
%\begin{subfigure}{0.45\textwidth}
%		\input{Kuvaajat/jäykistevoima.tex}
%		\label{fig:jäykistevoimakuvaaja}
%		\caption{}
%	\end{subfigure}
%	~
%	\begin{subfigure}{0.45\textwidth}
%		\input{Kuvaajat/jäykistevoima_2.tex}
%		\label{fig:jäykistevoimakuvaaja_2}
%		\caption{}
%	\end{subfigure}
%	\caption{Jäykisteen aksiaalisen jäykkyyden vaikutus jäykisteen puristusvoimaan kuvan (\ref{fig:puristettu_sauva}) mukaisessa rakenteessa.}
%\end{figure}
%
%Vasemmanpuoleisessa kuvaajassa (\ref{fig:jäykistevoimakuvaaja}) kuvataan jäykisteessä vaikuttavaa voimaa suhteessa jäykisteen aksiaaliseen jäykkyyteen. Kuvaajasta havaitaan, että jäykkyyden lähestyessä kaavassa (\ref{ideaalijäykkyys}) määritettyä ideaalisen jäykkyyden arvoa $\beta_i$, alkaa jäykisteessä vaikuttava puristusvoima kasvaa lähestyen ääretöntä. Oikeanpuoleisessa (\ref{fig:jäykistevoimakuvaaja_2}) kuvaajassa puolestaan havainnollistetaan jäykistevoiman suhdetta puristetun sauvan normaalivoimaan eri jäykkyysarvot omaavilla jäykisteillä. Kuvaajasta voidaan havaita, että jäykisteen jäykkyyden ollessa ainoastaan $2\beta_i$, on suurimmillaan jäykisteessä vaikuttava aksiaalinen voima 0,6 \% puristetussa sauvassa vaikuttavasta voimasta ja jäykkyydellä $4\beta_i$ n. 0,25 \%.
%
%Edellä esitettyjen kaavojen perusteella voidaan todeta, jäykisteeltä vaaditaan vähintään ideaalijäykkyyden (kaava \ref{ideaalijäykkyys}) mukaista aksiaalista jäykkyyttä, että alkuvinouden vaatimaa lujuutta ja että näillä kahdella on yhteys, joka on kuvan (\ref{fig:puristettu_sauva}) mukaisessa alkeisrakenteessa kaavan (\ref{jäykistevoima}) suuruinen.  

\subsection{Puristetun sauvan stabiliteetti EN 1993 mukaisesti}

Puristetun sauvan todellisen aksiaalisen puristuskapasiteetin määrittämiseksi on Eulerin teoreettisen nurjahduskapasiteetin lisäksi huomioitava epätarkkuustekijät, joita ovat esimerkiksi poikkeama ideaalisuorasta rakenteesta, materiaalin epälineaarisuus tai materiaalin muokkaamisen seurauksena syntyneet jäännösjännitykset. Näiden tekijöiden huomioiminen vaatii poikkeuksetta epälineaarista analyysia ja niiden laskentaan on historian aikana kehitetty laskentakaavoja, jotka perustuvat niin kokeelliseen tutkimukseen, kuin erilaisiin lujuusopin teorioihin.\parencite[27]{ziemian} Teräsrakenteiden suunnittelustandardi (\citeauthor{en1993})\todo{viittaus kuntoon!} esittää näiden poikkeamien huomioon ottamiseksi viisi (5) erilaista epätarkkuustekijää rakenteen valmistustavasta ja profiilin muodosta riippuen. Nämä epätarkkuustekijät on esitetty taulukossa (\ref{epätarkkuustekijät}). 

\begin{table}[htb]
\centering
\caption{Nurjahduskäyrien epätarkkuustekijät standardin (\citeauthor{en1993})  mukaan.}
\begin{tabular}{c c c c c c}
\label{epätarkkuustekijät}

Nurjahduskäyrä & $a_0$ & a & b & c & d \\
\hline
Epätarkkuustekijä $\alpha$ & 0,13 & 0,21 & 0,34 & 0,49 & 0,76
\end{tabular}
\end{table}

Epätarkkuustekijöiden perusteella standardiin on määritetty ja kuvaajin esitetty nurjahduskäyrät, jotka esittävät puristuskapasiteetin laskentaa varten tarvittavan pienennystekijän $\chi$ rakenteen hoikkuuden $\lambda$ funktiona. Kutakin epätarkkuustekijää vastaa oma nurjahduskäyränsä, ja ne on esitetty kuvaajassa (\ref{fig:nurjahduskäyräkuvaaja}). 

\begin{figure}[htb]
\input{Kuvaajat/nurjahdus.tex}
\caption{\citeauthor{en1993} mukaiset nurjahduskäyrät verrattuna Eulerin nurjahdukseen. }
\label{fig:nurjahduskäyräkuvaaja}
\end{figure}
\todo{symbolit käyrän labeleihin}

Standardin mukaan rakenteen nurjahduskapasiteetin $N_{b,Rd}$ ja plastisen puristuskapasiteetin $N_{pl,Rd}$ suhdetta kuvaa nurjahduksen pienennystekijä $\Phi$, joka määritetään kaavalla

\begin{equation}
\label{en_buckl_coeff}
\chi = \frac{N_{b,Rd}}{N_{pl,Rd}} =\frac{1}{\Phi + \sqrt{\Phi^2-\lambda^2}},
\end{equation}

missä $\lambda$ on poikkileikkauksen muunnettu hoikkuus ja $\chi$ epätarkkuuden huomioon ottava kerroin. Muunnettu hoikkuus määritetään kaavalla 

\begin{equation}
\label{hoikkuus}
\lambda = \sqrt{\frac{A f_y}{N_{cr}}},
\end{equation}

missä $N_{cr}$ Eulerin nurjahdusvoima (kaava \ref{euler}), $A$ on rakenteen poikkileikkauksen pinta-ala ja $f_y$ materiaalin myötölujuus. Mikäli rakenteen muunnettu hoikkuus täyttää ehdon $\lambda \leq 0,2$ ei nurjahdusta tarvitse standardin mukaan ottaa huomioon (kohta 6.3.1.2(4)). Edelleen kaavan (\ref{en_buckl_coeff}) termi $\Phi$ määritellään kaavalla

\begin{equation}
\label{phi}
\Phi = \frac{1+\alpha (\lambda-0.2)+\lambda^2}{2},
\end{equation}   

missä $\alpha$ on aiemmin mainuttu rakenteen muodosta ja valmistustavasta riippuva epätarkkuustekijä. 

Tässä diplomityössä käsiteltävät profiilit rajautuvat lujuusluokan S355 -rakenneputkiin, joiden nurjahduskäyrä on kylmämuovattuna c ($\alpha = 0,49$) ja kuumavalssattuna a ($\alpha = 0,21$).  



\subsection{Ristikon yläpaarteen tuenta hallin pituussuunnassa}

Teräsristikon puristettu yläpaarre on tuettava ristikon tasoon nähden poikittaisessa suunnassa rakennuksen pituussuuntaisin tukirakentein. Tukirakenteet on kuvattu ristikon tuentaa esittävässä kuvassa (\ref{fig:ristikon_nurjahdus}). Näistä tukirakenteista käytetään nimitystä kattositeet tai ristikon (nurjahdus)siteet ja ne kuuluvat osana rakennuksen jäykistysjärjestelmää.  \parencite{try_kaitila}
 
\begin{figure}[htb]

\includegraphics[width=15cm]{Kuvat/ristikko_upperchord_buckling.pdf}
\caption{Putkiristikon puristetun yläpaarteen nurjahdusmuoto ja nurjahdustuennat.}
\label{fig:ristikon_nurjahdus}
\end{figure}

Ristikon tuenta ja nurjahdustukien väliset etäisyydet $L$ suunnitellaan siten, että nurjahduksen kaavan (\ref{phi}) esittämä yläpaarteen nurjahduskapasiteetti $N_{b,Rd}$ ei alita paarteeseen ristikon pystykuormista aiheutuvaa aksiaalista puristusvoimaa. Puristusvoiman ylittäessä yläpaarteen nurjahduskapasiteetin, on yläpaarteeseen muodostuvan nurjahtavan sauvan nurjahdusmuoto kuvan (\ref{fig:ristikon_nurjahdus}) mukainen. Suunnittelustandardin (\citeauthor{en1993} liite BB) mukaan kaavassa (\ref{euler}) tarvittavaksi nurjahduspituudeksi voidaan putkiristikon paarteelle asettaa 0,9 kertaa ristikon poikittaistukien väli.

Ristikko kannattelee pystykuormaa palkin tavoin. Ristikon paarteet toimivat taivutuksessa kuten palkin laipat vastaanottaen taivutusmomentista aiheutuvan veto- ja puristusrasituksen. Uumasauvat välittävät leikkausrasituksia palkin uuman tavoin. Koska ristikko ei kuitenkaan ole jatkuva rakenne, paarteille kohdistuvan normaalivoiman jakauma on jatkuvan jakauman sijasta portaittainen. Tätä porrastusta on havainnollistettu kuvassa (\ref{fig:ristikon_yläpaarteen_puristus}), jossa on esitettynä ristikon yläpaarteen puristusvoiman porrastuksen periaate. Normaalivoima puristetulla yläpaarteella muttuu aina vedetyn uumasauvan kohdalla, kuten kuva esittää. 

\begin{figure}[htb]
\includegraphics[width=15cm]{Kuvat/ristikko_paarre_normaalivoima.pdf}
\caption{Periaatekuva normaalivoiman $N_c$ portaittaisesta jakautumisesta ristikon yläpaarteelle pystykuormasta $p$ Pratt -tyypin ristikossa.}
\label{fig:ristikon_yläpaarteen_puristus}
\end{figure}





%\section{Algoritmiavusteinen suunnittelu}
%\subsection{Parametrinen suunnittelu}
%\subsection{Algoritmiavusteinen mallinnus}
%\subsection{Teräsrakenteiden algoritmiavusteinen suunnittelu}

%\section{Teräshallin jäykistäminen}
%\subsection{Rakennuksen kokonaisstabiliteetti}
%\subsection{Teräshallin jäykistysjärjestelmän osat}
%\subsection{Standardit ja suunnitteluohjeet}

%\section{Pituussuuntaisen jäykistyksen optimointi Grasshopperilla}
%\subsection{Parametrit}
%\subsection{Suunnittelumuuttujat}
%\subsection{Ohjelman rakenne ja rajoitteet}
%\subsection{Optimointi Galapagoksella}
%\subsection{Tulokset}

%\section{Johtopäätökset}
%\subsection{Tavoitteiden saavuttaminen}
%\subsection{Tulosten analysointi}
%\subsection{Yhteenveto}


\section*{Viitteet}
\addcontentsline{toc}{section}{\protect\numberline{}Viitteet}
\renewcommand{\addcontentsline}[3]{}
\renewcommand{\section}[2]{}
%\bibliographystyle{authoryear} 
\printbibliography[title=none,heading=none]   % b) heading in Finnish
%\addtocontents{toc}{%               % b) add Finnish heading to table of contents
%\protect\noindent Lähteet\protect\par

\end{content}
\end{document}