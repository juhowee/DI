% Dokumentin asetukset
\documentclass[12pt]{article}

\usepackage{mathtools}

\usepackage[finnish,english]{babel}		% Kielimäärittelyt

\usepackage{lmodern}					% Tavutus skandinaavisille
\usepackage[T1]{fontenc}
\usepackage[utf8]{inputenc}

\usepackage{mathptmx}
	
\usepackage{tabularx}					% Paremmat taulukot

\usepackage{graphicx}					% Grafiikkapaketti

\usepackage{setspace} 					% Kansilehden tarkastajaboksia varten
\usepackage{hyphenat} 					% Joku höskä kansilehden toimintaa varten


%Lähdeluettelon muotoilua
% Määritetään viittaukset nimi-vuosiluku -tyylille
\usepackage[backend=biber,style=authoryear,giveninits=true,maxcitenames=2]{biblatex}		
% Pakotetaan lähdeluettelon muotoilu sukunimi-etunimi -muotoon				
\DeclareNameAlias{author}{last-first}
% Lähdeluettelon merkintöjen välille 1.5 riviväliä
\setlength\bibitemsep{1.5\itemsep}
% Poistetaan lainausmerkit "" ja kursiivit teoksen nimistä lähdeluettelossa
\DeclareFieldFormat*{title}{#1}
\DeclareFieldFormat*{journaltitle}{#1}
\DeclareFieldFormat*{booktitle}{#1}

% Korvataan "ja" -> & -merkillä lähdeluettelossa
\AtBeginBibliography{
\renewcommand*{\finalnamedelim}{%
\ifnumgreater{\value{liststop}}{2}{}{}%
\addspace\&\space}%
}

% Korvataan "ja" -> & -merkillä viitteissä
\AtEveryCite{
\renewcommand*{\finalnamedelim}{%
\ifnumgreater{\value{liststop}}{2}{}{}%
\addspace\&\space}%
}

\usepackage[parfill]{parskip}			% Poistaa kappaleen ensimmäisen rivin sisennyksen käytöstä

\usepackage{helvet}						% Helvetica -fontti otsikoita varten

\usepackage{titlesec}					% Otsikkojen tyylien muokkaus
\usepackage[ampersand]{easylist}
\usepackage{amssymb}

\usepackage{fancyhdr}

\usepackage[textsize=small]{todonotes}	% TODO -merkinnät
\setlength{\marginparwidth}{2.3cm}

% Kuva- ja taulukkotekstien asetukset
\usepackage[labelsep=period]{caption}
\captionsetup{labelfont={it,bf},textfont=it}

% Listausmerkinnät
\ListProperties(Hide=100, Space*=0.6pt, Hang=true, Progressive=3ex, Style*=-- ,
Style2*=$\bullet$ ,Style3*=$\circ$ ,Style4*=\tiny$\blacksquare$ )

% Sivun asetukset
\RequirePackage[paper=a4paper,inner=4cm,outer=2cm,top=2.5cm,bottom=2.5cm,headsep=1em]{geometry}
%\setlength\brokenpenalty{1000}
\setstretch{1.2} % Line height

% Asetetaan uusi otsikko (section) alkamaan uudelta sivulta
\newcommand{\sectionbreak}{\clearpage}

% Sivunumerointi sivun oikeaan yläkulmaan
\pagestyle{myheadings}

% Sivunumerointityyli precontent -osiossa (roomalaiset)
\newenvironment{precontent}{\pagenumbering{roman}}{\cleardoublepage}

% Sivunumerointityyli content -osiossa (numeerinen)
\newenvironment{content}{\pagenumbering{arabic}}{}

% Sivunumerointi vain erillisellä komennolla 
\pagenumbering{gobble}

% Dokumentin tiedot
\title{TERÄSRAKENTEISEN HALLIN PITUUSSUUNTAISTEN JÄYKISTERAKENTEIDEN OPTIMOINTI}
\author{JUHO VASTAPUU}
\newcommand{\documenttype}{Diplomityö} % Työn tyyppi kansilehdellä, esim. "diplomityö"
\newcommand{\inspector}   {TkT Kristo Mela } % Työn tarkastaja
\newcommand{\approvaldate}{6.6.2016}      % Työn hyväksyttämispäivämäärä

% Sisällysluettelon otsikko suomeksi.
\addto\captionsfinnish{
  \renewcommand{\contentsname}%
    {SISÄLLYSLUETTELO}%
}

% ------------
% Images
%
% The first parameter is width of the image, and defaults to
% 70% of page width.
% Usage example: \centeredpicture[1.0]{esimKuva}{Matlabilla tehty PDF-muotoinen esimerkkikuva.}
% ------------
\newcommand{\centeredpicture}[3][0.7]{
        \begin{figure}[h]
        \begin{center}
        \includegraphics[width=#1\textwidth]{#2}
        \end{center}
        \caption{#3}
        \label{fig:#2}
        \end{figure}
}

\newcommand{\tablefromimage}[3][0.7]{
        \begin{table}[h]
        \caption{#3}
        \label{table:#2}
        \end{table}
        \begin{figure}[h]
        \begin{center}
        \includegraphics[width=#1\textwidth]{#2}
        \end{center}
        \end{figure}
}

% Poistaa () -merkinnät vuosiluvun ympäriltä lähdeluettelossa
%\renewcommand\harvardyearleft{\unskip, }
%\renewcommand\harvardyearright[1]{.} 



\addbibresource{biblio.bib}



\begin{document}

\selectlanguage{finnish}


\include{kansilehti}

% Määritetään precontent -osion otsikkotyyli
\titleformat{\section}
        {\Large\sffamily\bfseries}
        {\makebox[2em][l]{\thesection}}
        {0pt}
        {\uppercase}
\titlespacing{\section}{0pt}{0pt}{18pt}

\begin{precontent}
\section*{Tiivistelmä}
\begin{spacing}{1.0}
{\fontfamily{phv}\selectfont
\textbf{JUHO VASTAPUU}: Teräsrakenteisen hallin pituussuuntaisen jäykistysjärjestelmän optimointi\\
Tampereen teknillinen yliopisto\\
Diplomityö, x sivua, x liitesivua\\
Joulukuu 2018\\
Rakennustekniikan diplomi-insinöörin tutkinto-ohjelma\\
Pääaine: Rakennesuunnittelu\\
Tarkastaja: TkT Kristo Mela\par

Avainsanat: \par
}
\end{spacing}





\section*{Alkusanat}


\tableofcontents

\end{precontent}

% Määritetään content -osion otsikkotyylit
\titleformat{\section}
        {\Large\sffamily\bfseries}
        {\makebox[2em][l]{\thesection}}
        {0pt}
        {\uppercase}
\titlespacing{\section}{0pt}{0pt}{42pt}

\titleformat{\subsection}
        {\large\sffamily\bfseries}
        {\makebox[2em][l]{\thesubsection}}
        {0pt}
        {}
\titlespacing{\subsection}{0pt}{18pt}{12pt}

\titleformat{\subsubsection}
        {\large\sffamily\bfseries}
        {\makebox[3em][l]{\thesubsubsection}}
        {0pt}
        {}
\titlespacing{\subsubsection}{0pt}{18pt}{12pt}



\begin{content}


\section{Johdanto}

\subsection{Tutkimuksen tausta}
\subsection{Työn tavoitteet}
\subsection{Työn rajaukset}

\section{Optimointi}
\subsection{Optimointitehtävän matemaattinen määrittely}

Matemaattisella optimoinnilla tarkoitetaan prosessia, jolla löydetään jollekkin funktiolle paras mahdollinen arvo sille asetetut reunaehdot huomioiden. Asettamalla optimoitava kohde sekä halutut rajoite-ehdot matemaattiseen muotoon, voidaan optimoimalla löytää matemaattisin keinoin paras käypä ratkaisu. Käyvällä ratkaisulla tarkoitetaan ratkaisua, joka kuuluu annettujen rajoite-ehtojen joukkoon. 

Matemaattisesti optimoinnissa on tavoitteena etsiä funktiolle käyvästä joukosta minimi- tai maksimiarvo. Optimointitekniikoita ja algoritmeja on kehitetty lukuisia ja kukin niistä soveltuu käytettäväksi eri tavalla eri optimointitehtäviin. Optimointi ja erilaisten optimoitimenetelmien tutkiminen on yksi matemaattisen operaatiotutkimuksen osa-alueista. Optimoinnista voidaan joissain yhteyksissä käyttää myös nimitystä matemaattinen ohjelmointi (mathematical programming), jolla viitataan matemaattisten algoritmien kehittämiseen ja ohjelmoimista optimointitarkoituksiin.  \parencite[1]{rao}

Optimointitehtävä kirjoitetaan matemaattisesti seuraavanlaisessa muodossa.
\begin{equation*}
\text{Etsi } \textbf{x} = 
\begin{Bmatrix} 
x_1 \\ 
x_2 \\ 
\vdots \\
x_n  
\end{Bmatrix}
\text{    joka minimoi } f(\textbf{x}) \text{, siten että}
\end{equation*}
\begin{align}
\label{rajoitteet}
\begin{split}
g_i(\textbf{x}) \leq 0, \quad j = 1,2, \cdots , m  \\ 
h_j(\textbf{x}) = 0, \quad j = 1,2, \cdots , p
\end{split}
\end{align}

missä \textbf{x} on vektori, joka sisältää n-kappaletta suunnittelumuuttujia, f(\textbf{x}) on tavoitefunktio, $g_i(\textbf{x})$ ja $h_j(\textbf{x})$ ovat rajoite-ehtoja. Rajoite-ehdot voivat olla joko epäyhtälö- tai yhtälömuotoisesti ilmoitettuja. Suunnittelumuuttujien lukumäärä (n) sekä rajoite-ehtojen lukumäärä (m ja/tai p) eivät ole riippuvaisia toisistaan. Tällaista optimointitehtävää kutsutaan rajoitetuksi optimointiongelmaksi. Optimointiongelman ei kuitenkaan tarvitse olla rajoitettu, vaan se voidaan ilmoittaa myös rajoittamattomana. Kaavassa \eqref{rajoitteet} on esitetty optimointitehtävän standardimuotoinen asettelu (standard design optimization model). \parencite[6]{rao}

Vektori \textbf{x} sisältää optimointitehtävän kaikki suunnittelumuuttujat (design variables). Muuttamalla jonkin suunnittelumuuttujan $x_i$ arvoa, muuttuu myös tavoitefunktion f(\textbf{x}) arvo. Suunnittelumuuttujista voidaan käyttää myös nimitystä optimointimuuttujat tai vapaat muuttujat, eli niiden arvoja voidaan muutella vapaasti kun haetaan tavoitefunktiolle arvoa. Toisistaan riippumattomien eli itsenäisten suunnittelumuuttujien lukumäärä on optimointiongelman vapausasteluku (design degree of freedom). Yleisesti ottaen suunnittelumuuttujien tulee olla toisistaan riippumattomia, mutta joissain tapauksissa niiden määrä voi olla ongelman vapausastelukua suurempi. Tämä on perusteltua esimerkiksi silloin, kun kohdefunktion määrittely pelkillä itsenäisillä suunnittelumuuttujilla olisi hankalaa. Jokaiselle suunnittelumuuttujalle täytyy myös pystyä asettamaan jokin numeerinen lähtöarvo, jotta optimointitehtävä pystytään suorittamaan. 

Kohde- tai tavoitefunktio f(\textbf{x}) (objective function) on optimointitehtävän matemaattinen muoto ilmoitettuna suunnittelumuuttujavektorin \textbf{x} funktiona. Optimointitehtävän tavoitteena on joko minimoida tai maksimoida kohdefunktion arvo. Mikäli optimointitehtävässä on useampi kuin yksi kohdefunktio, käytetään tehtävästä nimitystä monitavoiteoptimointi (multiobjective design optimization). Tällöin kohdefunktio ilmaistaan matemaattisesti kohdefunktioiden joukkona 
\begin{align}
\textbf{f(x)} = \begin{bmatrix}
f_1(\textbf{x}) &  f_2(\textbf{x}) & \cdots & f_p(\textbf{x})
\end{bmatrix},
\end{align}
jossa jokainen kohdefunktio koostuu kuitenkin samasta suunnittelumuuttujavektorista \textbf{x}.

Optimoitavalle kohteelle asetettavat rajoite-ehdot esitetään rajoitefunktioina $g_i(\textbf{x})$ ja $h_j(\textbf{x})$. Optimointialgoritmi ratkaisee optimointitehtävän siten, että kohdefunktion arvo toteuttaa rajoite-ehdot. Rajoite-ehtojen muodostamaa joukkoa kutsutaan täten optimointiongelman käyväksi joukoksi (feasible region). Mitä tahansa käyvässä joukossa olevaa kohdefunktion arvoa kutsutaan käyväksi ratkaisuksi (feasible design) huolimatta siitä onko kyseessä optimiratkaisu. Kaksiuloitteisessa tapauksessa käypää joukkoa voidaan havainnollistaa piirtämällä rajoitefunktioiden käyrät koordinaatistoon. Käypä joukko muodostuu näiden käyrien rajoittamana alueena. Käyvän joukon negaatiota kutsutaan ei-käyväksi joukoksi (unfeasible region). Sekä tavoitefunktion f(\textbf{x}), että rajoite-ehtojen $g_i(\textbf{x})$ ja $h_j(\textbf{x})$ on oltava toisitaan joko implisiittisesti tai eksplisiittisesti riippuvia. Mikäli riippuvuutta funktioiden välillä ei ole, ei optimointitehtävää voi muodostaa eikä varsinaista optimointiongelmaa voi edes osoittaa. \parencite[43]{arora} 

Kuten optimointitehtävän määrittelevä kaava \eqref{rajoitteet} osoittaa, optimointiongelmalle voidaan asettaa rajoite-ehtoja sekä yhtälö- että epäyhtälömuodossa. Epäyhtälörajoitteita kutsutaan käypään joukkoon nähden toispuoleisiksi rajoite-ehdoiksi (unilateral constraints tai one-sided constraints). Epäyhtälörajoitteiden rajoittama käypä joukko on täten paljon laajempi kun verrataan yhtälörajoitteista käypää joukkoa. Esimerkiksi kakisuloitteisessa tapauksessa yhtälörajoite tarkoittaisi, että käypä ratkaisu löytyisi rajoitefunktion käyrältä. \parencite[16-18]{arora} 

Yhtälömuotoisten rajoite-ehtojen määrän tulee olla maksimissaan suunnittelumuuttujien määrä, toisin sanottuna optimointitehtävän kaavan \eqref{rajoitteet} tulee toteuttaa ehto

\begin{equation}
\label{yhtälörajoite-ehto}
p \leq n.
\end{equation}

Tapaus, jossa yhtälömuotoisia rajoite-ehtoja on annettuja suunnittelumuuttujia enemmän, on kyseessä ylimääritetty yhtälöryhmä (overdetermined system). Tällaisessa tapauksessa rajoite-ehtojen joukossa on ylimääräisiä eli redundatteja ehtoja, jotka toteuttavat suoraan jonkun muun rajoite-ehdon, eikä niiden ilmoittaminen täten ole tarpeellista. Triviaalitapauksessa jossa suunnittelumuuttujien määrä ja yhtälömuotoisten rajoite-ehtojen määrä on yhtäsuuri, löytyy tehtävälle ratkaisu analyyttisin keinoin eikä optimointi ole tarpeellista. Kaksiuloitteisessa tapauksessa tämä tarkoittaisi kahden käyrän leikkauspistettä. 

Standardimuotoisessa optimointitehtävässä epäyhtälörajoitteet ilmoitetaan aina kaavan \eqref{rajoitteet} osoittamassa muodossa, eli siten että rajoite-ehto on pienempi tai yhtäsuuri kuin nolla ($\leq$ 0). Tästä huolimatta voidaan optimointitehtävässä käsitellä myös $\geq$ -tyyppisiä rajoite-ehtoja. Standardimuotoista tehtävää aseteltaessa nämä voidaan muuttaa $\leq$ -muotoon yksinkertaisesti kertomalla ehto luvulla -1. Epäyhtälömuotoisten rajoite-ehtojen määrää ei ole rajoitettu, toisin kuin yhtälömuotoiset rajoitteet kaavassa \eqref{yhtälörajoite-ehto}. Niiden määrää ei ole siis rajoitettu.   \parencite[43]{arora}

\subsection{Optimointitehtävän luokittelu}
\todo{Tähän alustus}
\begin{enumerate}
\item Rajoitetuksi tai ei-rajoitetuksi
\item Staattiseksi tai dynaamiseksi
\item Kohdefunktion tai rajoite-ehtojen matemaattisen muodon perusteella
\item Säätöongelmaksi
\item Diskreetiksi tai jatkuvaksi tehtäväksi
\item Deterministiseksi tai stokastiseksi tehtäväksi
\item Kohdefunktion separoituvuuden perusteella
\item Kohdefunktioiden määrän perusteella
\end{enumerate}

\parencite{rao}

Rajoitefunktioiden perusteella tehtävä voidaan luokitella joko rajoitetuksi- tai ei-rajoitetuksi tehtäväksi. Mikäli tehtävällä on yksikin rajoitefunktio, on kyseessä rajoitettu optimointitehtävä. Luokittelua voidaan tarkentaa edelleen osittain rajoitetuksi tai täysin suljetuksi systeemiksi. Suljetulla systeemillä tarkoitetaan tilannetta, jossa rajoitefunktiot muodostavat äärellisen kokoisen käyvän joukon. 

Suunnittelumuuttujien perusteella tehtävä voidaan jakaa staattiseksi tai dynaamiseksi tehtäväksi. Staattisessa tai parametrisessa tehtävässä kohdefunktio on määritelty suunnittelumuuttujien suhteessa ja tehtävänä on ratkaista suunnittelumuuttujat. Kaavan \eqref{rajoitteet} määrittelyssä kyseessä on staattinen optimointitehtävä, jossa siis etsitään suunnittelumuuttujille arvo siten, että se minimoi kohdefunktion. Dynaamisessa tehtävässä kohdefunktio puolestaan muodostuu funktioista, jotka on määritelty jonkun tietyn parametrin suhteen, kuten esimerkiksi seuraavasti.

\begin{equation}
\label{dynaaminen}
\text{Etsi } \textbf{x}(t) = 
\begin{Bmatrix} 
x_1(t) \\ 
x_2(t) \\ 
\vdots \\ 
x_n(t) \\ 
\end{Bmatrix}
\text{    joka minimoi } f[\textbf{x}(t)].
\end{equation}

Dynaamisessa tehtävässä etsitään siis kohdefunktioon sijoitettavien suunnittelumuuttujien sijasta funktiot, jotka esitetään jonkin parametrin suhteen. \parencite[15]{rao}

Optimointitehtävä luokitellaan myös kohdefunktion tai rajoite-ehtojen matemaattisen muodon perusteella. Tämä luokittelutapa on erityisen kriittinen optimointitehtävän ratkaisun kannalta, sillä monet optimointialgoritmit toimivat vain tietyntyyppisille optimointitehtäville juurikin kohdefunktioiden tai rajoite-ehtojen matemaattisen muodon mukaan. Yksi yleisin optimointitehtävän muoto on epälineaarinen ongelma (nonlinear programming problem, NLP). Optimointitehtävä on epälineaarinen mikäli sen kohdefunktio tai yksikin rajoitefunktioista on muodoltaan epälineaarinen. Optimoinnissa muodostuvat ongelmat ovat hyvin usein muodoltaan epälineaarisia ja matematiikan osa-aluetta, joka tutkii epälineaaristen optimointiongelmien ratkaisua, kutsutaan epälineaariseksi matemaattiseksi ohjelmoinniksi (nonlinear programming). Muita optimoinnin tehtävätyyppejä ovat esimerkiksi geometrinen ja kvadraattinen ongelma. \parencite[19]{rao} 

\todo{tarkista oikea suomenkielinen termi !}
Säätöongelmassa tai säätöoptimointitehtävässä (optimal control) on kyse tehtävästä, joka jakautuu useampiin osioihin tai sekvensseihin \todo{tarkista termin oikeellisuus}. Jokainen sekvenssi kehittyy edellisestään määritetyllä tavalla. Tehtävän määrittelyyn käytetään suunnittelumuuttujien lisäksi tilamuuttujia (state variables). Säätöoptimointitehtävässä suunnittelumuuttujat määrittävät systeemin kussakin sekvenssissä sekä tavan, jolla systeemi siirtyy seuraavaan sekvenssiin. Tilamuuttujat puolestaan määrittävät kussakin sekvenssissä systeemin tilan, eli tutkittavan ongelman käyttäytymisen kussakin sekvenssissä. Säätöongelmassa tehtävänä on löytää suunnittelu- tai tilamuuttujille sellaiset arvot kussakin eri sekvenssissä, että kohdefunktioiden summa eri sekvensseissä saadaan minimoitua rajoite-ehdot huomoiden. \parencite[19]{rao} Tämänkaltaiset tehtävät ovat tavallisia sellaisissa teknisissä sovelluksissa, joiden tila muuttuu jatkuvasti ja tilan ylläpitoon vaaditaan resursseja. Yksi yleinen säätöoptimointitehtäviä soveltava tekniikan ala on säätötekniikka.

Suunnittelumuuttujien saadessa vain diskreettejä arvoja, käytetään optimointitehtävästä nimitystä diskreetti tai ei-jatkuva tehtävä. Tämän kaltainen tehtävä voidaan yleistää kokonaislukuoptimoinniksi (integer programming problem). Tehtävän vastakohta on jatkuva tehtävä, jossa siis sallitaan kaikille suunnittelumuuttujille reaalilukuarvo (real-valued programming problem). \parencite[28]{rao} Insinööritieteissä suunnittelumuuttujat valitaan yleensä ennalta määritellystä joukosta käytössä olevien resurssien mukaan, eli käsiteltävät optimointitehtävät ovat usein diskreettejä. 

Suunnittelumuuttujat tai kohdefunktion parametrit voivat saada määriteltyjen eli deterministisien arvojen sijasta todennäköisyyteen perustuvia arvoja. Tällöin optimointitehtävä on stokastinen eli siinä käsitellään determinististen muuttujien sijasta satunnaismuuttujia. \parencite[29]{rao} \todo{tähän lisää}

\subsection{Rakenteiden optimointi}


\section{Teräshallin jäykistäminen}
\subsection{Puristetun sauvan stabiliteetti}
Puristetun sauvan stabiliteetti perustuu Eulerin nurjahdukseen, jossa ideaalisuoraa homogeenisesta materiaalista koostuvaa hoikkaa pilaria kuormitetaan keskeisesti. Voimaa, joka aiheuttaa pilarin nurjahduksen kutsutaan nurjahduskuormaksi tai kriittiseksi voimaksi ja se määritellään kaavalla

\begin{equation}
N_{cr} = \frac{\pi^2 E I}{L^2},
\end{equation}

missä E on materiaalin kimmokerroin, I on poikkileikkauksen neliömomentti tarkasteltavan akselin ympäri ja L on pilarin kyseeseen tulevaa nurjahdusmuotoa nurjahduspituus. 

Standardin  \parencite{en1993}

\subsection{Ristikon yläpaarteen tuenta}
Teräsristikon puristettu yläpaarre on tuettava ristikkon tasoon nähden poikittaisessa suunnassa tukirakenteella, joka on toisesta päästään kiinnittetty sivusiirtymättömään rakenteeseen. Nämä sivuttaiset tuet kuuluvat osana katon jäykistysjärjestelmää. 
 











%\section{Algoritmiavusteinen suunnittelu}
%\subsection{Parametrinen suunnittelu}
%\subsection{Algoritmiavusteinen mallinnus}
%\subsection{Teräsrakenteiden algoritmiavusteinen suunnittelu}

%\section{Teräshallin jäykistäminen}
%\subsection{Rakennuksen kokonaisstabiliteetti}
%\subsection{Teräshallin jäykistysjärjestelmän osat}
%\subsection{Standardit ja suunnitteluohjeet}

%\section{Pituussuuntaisen jäykistyksen optimointi Grasshopperilla}
%\subsection{Parametrit}
%\subsection{Suunnittelumuuttujat}
%\subsection{Ohjelman rakenne ja rajoitteet}
%\subsection{Optimointi Galapagoksella}
%\subsection{Tulokset}

%\section{Johtopäätökset}
%\subsection{Tavoitteiden saavuttaminen}
%\subsection{Tulosten analysointi}
%\subsection{Yhteenveto}


\section*{Viitteet}
\addcontentsline{toc}{section}{\protect\numberline{}Viitteet}
\renewcommand{\addcontentsline}[3]{}
\renewcommand{\section}[2]{}
%\bibliographystyle{authoryear} 
\printbibliography[title=none,heading=none]   % b) heading in Finnish
%\addtocontents{toc}{%               % b) add Finnish heading to table of contents
%\protect\noindent Lähteet\protect\par

\end{content}
\end{document}